%-------------------------------------------------------------------------------
% COMPILATION
% pdflatex -shell-escape <fichier.tex>
%-------------------------------------------------------------------------------

\documentclass[a4paper]{article}

%% Language and font encodings
\usepackage[frenchb]{babel}
\usepackage[utf8x]{inputenc}
\usepackage[T1]{fontenc}
\usepackage{minted} %compiler avec la commande -shell-escape
\usepackage{graphicx}

\usepackage{enumitem,amssymb}
\newlist{todolist}{itemize}{2}
\setlist[todolist]{label=$\square$}
\usepackage{pifont}
\newcommand{\cmark}{\ding{51}}%
\newcommand{\xmark}{\ding{55}}%
\newcommand{\done}{\rlap{$\square$}{\raisebox{2pt}{\large\hspace{1pt}\cmark}}%
\hspace{-2.5pt}}
\newcommand{\wontfix}{\rlap{$\square$}{\large\hspace{1pt}\xmark}}
\newcommand*{\escape}[1]{\texttt{\textbackslash#1}}
\newcommand*{\escapeI}[1]{\texttt{\expandafter\string\csname #1\endcsname}}
\newcommand*{\escapeII}[1]{\texttt{\char`\\#1}}

%% Sets page size and margins
\usepackage[a4paper,top=3cm,bottom=2cm,left=3cm,right=3cm,marginparwidth=1.75cm]{geometry}
\setlength{\parskip}{.5em}

\newcommand{\HRule}{\rule{\linewidth}{0.5mm}}

%-------------------------------------------------------------------------------
% TITLE PAGE
%-------------------------------------------------------------------------------

\title
{
	\LARGE{Projet Approche Objet}
	\HRule \\ [0.5cm]
	\LARGE \textbf{\uppercase{Un jeu de plateau Labyrinthe}}
	\HRule \\ [0.5cm]
}

\author{Guillaume CHARLET \\ Kenji FONTAINE \\ Romain ORDONEZ \\ Adrien HALNAUT}

\begin{document}

\null  % Empty line
\nointerlineskip  % No skip for prev line
\vfill
\let\snewpage \newpage
\let\newpage \relax
\maketitle
\let \newpage \snewpage
\vfill
\break % page break

%-------------------------------------------------------------------------------
% Table of Contents
%-------------------------------------------------------------------------------

\tableofcontents
\newpage

%-------------------------------------------------------------------------------
% Introduction
%-------------------------------------------------------------------------------

\section{Description du projet}
Ce travail a été réalisé dans un cadre universitaire,  par des étudiants en
Master 1 informatique à l'université de Bordeaux.
Le but de ce projet est de mettre en pratique les connaissances acquises lors
du cours d'Approche Objet. \\
Labyrinth est un jeu de plateau dans lequel le joueur peut se déplacer et devra
trouver la sortie dans un labyrinthe en franchissant divers obstacles.

%-------------------------------------------------------------------------------
% Description du travail
%-------------------------------------------------------------------------------

\section{Description du travail}

\subsection{Première séance}
Lors de la première séance, nous avons réfléchi sur la structure qu'allait
prendre notre code. Nous nous sommes mis d'accord pour partir sur une
architecture MVC (modèle, vue et contrôleur) afin de disposer d'une certaine
flexibilité. Nous avons ensuite déterminer sur papier, la hiérarchie de nos
classes, packages, etc... Puis nous nous sommes réparti le travail en groupe. \\
Voici le premier modèle retenu :
\begin{itemize}

	\item Model \begin{itemize}
		\item Drawable \begin{itemize}
			\item Gentil
			\item Mechant
		\end{itemize}
		\item Static \begin{itemize}
			\item Switches
			\item Doors
			\item Bonus
		\end{itemize}
		\item Game core \begin{itemize}
			\item Laby/Graph
			\item Menu
		\end{itemize}
	\end{itemize}

	\item View \begin{itemize}
		\item Rien pour l'instant
	\end{itemize}

	\item Controller \begin{itemize}
		\item Main/Master
		\item In-Game \begin{itemize}
			\item Inputs
			\item Events
			\item GameManager
		\end{itemize}
	\end{itemize}

\end{itemize}

\subsection{Premier pas}
Nous avons commencé notre implémentation par la génération de graphe (modèle)
ainsi que l'affichage de la grille (vue).

%-------------------------------------------------------------------------------
%
%-------------------------------------------------------------------------------

%-------------------------------------------------------------------------------
%
%-------------------------------------------------------------------------------

%-------------------------------------------------------------------------------
% END
%-------------------------------------------------------------------------------

\end{document}
