%-------------------------------------------------------------------------------
% COMPILATION
% pdflatex -shell-escape <fichier.tex>
%-------------------------------------------------------------------------------

\documentclass[a4paper]{article}

%% Language and font encodings
\usepackage[frenchb]{babel}
\usepackage[utf8x]{inputenc}
\usepackage[T1]{fontenc}
\usepackage{minted} %compiler avec la commande -shell-escape
\usepackage{graphicx}
\usepackage{titlesec}
\usepackage{hyperref}

%% Sets page size and margins
\usepackage[a4paper,top=3cm,bottom=2cm,left=3cm,right=3cm,marginparwidth=1.75cm]{geometry}
\setlength{\parskip}{.5em}

\newcommand{\HRule}{\rule{\linewidth}{0.5mm}}

%-------------------------------------------------------------------------------
% TITLE PAGE
%-------------------------------------------------------------------------------

\title
{
	\LARGE{Projet Approche Objet}
	\HRule \\ [0.5cm]
	\LARGE \textbf{\uppercase{Un jeu de plateau Labyrinthe}}
	\HRule \\ [0.5cm]
}

\author{Guillaume CHARLET \\ Kenji FONTAINE \\ Romain ORDONEZ \\ Adrien HALNAUT}

\begin{document}

\null  % Empty line
\nointerlineskip  % No skip for prev line
\vfill
\let\snewpage \newpage
\let\newpage \relax
\maketitle
\let \newpage \snewpage
\vfill
\break % page break

%-------------------------------------------------------------------------------
% Table of Contents
%-------------------------------------------------------------------------------

\tableofcontents
\newpage

%-------------------------------------------------------------------------------
% Introduction
%-------------------------------------------------------------------------------

\section{Description du projet}
Ce travail a été réalisé dans un cadre universitaire,  par des étudiants en
Master 1 informatique à l'université de Bordeaux.
Le but de ce projet est de mettre en pratique les connaissances acquises lors
du cours d'Approche Objet. \\
Labyrinth est un jeu de plateau dans lequel le joueur peut se déplacer et devra
trouver la sortie dans un labyrinthe en franchissant divers obstacles.

%-------------------------------------------------------------------------------
% Architecture du code
%-------------------------------------------------------------------------------

\section{Architecture du code}

Le dossier src contient notre code source pour le projet, réparti dans
différents sous-dossiers. On y trouve également Main, qui est éxécuté lorsque
le programme est lancé.

\subsection{Main}
Description de Main

\subsubsection{controllers}
\textbf{Master :} Description de Master.

\textbf{ingame :} \begin{itemize}
	\item{EntitySpawner : Instancie les entités du dossier drawable
	avec une position aléatoire.}
	\item{EventManager : S'occupe de détecter et de gérer les collisions entre
	entités (entre le joueur et un ennemi par exemple).}
	\item{GameManager : Gère le déroulement du jeu. Initialisation, gestion des
	déplacements des entités, ainsi que leur destruction (si un bonus est
	ramassé par exemple).}
	\item{Inputs : Reçoit les entrées utilisateurs afin de déplacer le joueur
	dans le labyrinthe.}
	\item{SpriteManager : Permet de récupérer les images utilisées dans
	l'interface graphique.}
\end{itemize}

\subsubsection{models}
\textbf{drawable :} \begin{itemize}
	\item{Bonus : Classe des bonbons ramassables par le joueur.}
	\item{Character : Classe utilisée par le joueur ainsi que les ennemis. On
	distingue le joueur des ennemis à l'aide d'un booléen.}
	\item{Door : Classe de la porte de sortie utilisée dans le labyrinthe.}
	\item{Entity : Classe abstraite héritée par toutes les autres entités. Elle
	définie le fonctionnement générale d'une entité.}
	\item{EntityType : Enumération des différentes entités pouvant être
	utilisées dans le jeu.}
	\item{OnOff : Classe des boutons permettant d'activer/désactiver un mur dans
	lebyrinthe.}
	\item{SpriteType : Enumérations des différentes images pouvant être
	associées à une entité dans le jeu.}
\end{itemize}

\textbf{game :} \begin{itemize}
	\item{maze/Direction : Enumération des directions différentes qu'un
	mouvement peut prendre.}
	\item{maze/Menu : Classe du Menu, non utilisée.}
	\item{maze/WallType : Enumération des différents états qu'un mur peut
	prendre.}
\end{itemize}

\subsubsection{views}
\textbf{ViewFrame :} Permet d'initialiser l'interface graphique ainsi que de
mettre l'affichage des entités à jour à chaque déplacement.

%-------------------------------------------------------------------------------
% Description du travail
%-------------------------------------------------------------------------------

\section*{} % Sinon compile pas
\section{Description du travail}

\subsection{Première séance}
Lors de la première séance, nous avons réfléchi sur la structure qu'allait
prendre notre code. Nous nous sommes mis d'accord pour partir sur une
architecture MVC (modèle, vue et contrôleur) afin de disposer d'une certaine
flexibilité. Nous avons ensuite déterminer sur papier, la hiérarchie de nos
classes, packages, etc... Puis nous nous sommes réparti le travail en groupe. \\
Voici le premier modèle retenu :
\begin{itemize}

	\item Model \begin{itemize}
		\item Drawable \begin{itemize}
			\item Gentil
			\item Mechant
		\end{itemize}
		\item Static \begin{itemize}
			\item Switches
			\item Doors
			\item Bonus
		\end{itemize}
		\item Game core \begin{itemize}
			\item Laby/Graph
			\item Menu
		\end{itemize}
	\end{itemize}

	\item View \begin{itemize}
		\item Rien pour l'instant
	\end{itemize}

	\item Controller \begin{itemize}
		\item Main/Master
		\item In-Game \begin{itemize}
			\item Inputs
			\item Events
			\item GameManager
		\end{itemize}
	\end{itemize}

\end{itemize}

\subsection{Premier pas}
Nous avons commencé notre implémentation par la génération de graphe (modèle)
ainsi que l'affichage de la grille (vue). Le but était d'obtenir une interface
graphique avec un affichage d'un labyrinthe. Pour cela, il nous a également
fallu implémenter Master (contrôleur) afin de faire le lien entre le graphe
généré en modèle et l'affichage. Cette partie nous a pas posé de
problème particulier. Guillaume s'est occupé de la partie contrôleur, Adrien
de la génération de graphe et l'affichage a été fait par Romain et Kenji.

\subsubsection{Modèle graphe}

Nous avons préféré implémenter notre propre module de graphes, celui-ci se rapproche de celui proposé dans le sujet, mais en un peu plus simpliste et plus destiné à une utilisation pour un jeu de plateau. Ce module est abstrait du reste du code par la classe Maze, qui s'occupe des manipulations moins atomiques sur le graphe. Ainsi, la classe Graphe et ses dépendances gèrent la manipulation de graphe, c'est-à-dire en utilisant des points (Vertex) et des liaisons (Edge) en tant que tel, et la classe Maze, utilisée par notre jeu, gère les transformations et la récupération d'informations de Graphe. Adrien s'est occupé de son implémentation principale, et Romain ainsi que Kenji de modeler son utilisation pour remplir les objectifs du jeu (respectivement calcul de parcours/distance, gestion des portes et interrupteurs).

\subsection{Implémentation des entités}

\subsubsection{Déplacement du joueur}
Une fois la génération et l'affichage du labyrinthe effectués, notre prochain
objectif était d'implémenter le joueur (affichage + déplacement).
Pour se faire, nous avons commencé par implémenter une classe abstraite Entity,
afin de ne pas avoir de redondance dans notre code.
Puis nous avons crée une classe Character héritant de Entity, en implémentant
les quelques fonctionnalités propres à cette classe (comme le déplacement).
Cette partie a principalement été faite par Adrien.

\subsubsection{Objets statiques}
L'implémentation des objets statiques que l'on peut afficher tel que les
bonbons a été faite de façon naturelle. Il nous suffisait d'hériter de la
classe Entity et d'implémenter les quelques fonctionnalités propres à chaque
objet. Sauf pour les boutons OnOff permettant d'activer/désactiver une porte,
étant lié à un certain mur, il était plus compliqué d'implémenter ces boutons.
Tous les membres ont contribués à l'implémentation des différents objets.

\subsubsection{Gestion des collisions}
Le prochain objectif était de faire disparaitre les bonbons au contact du
joueur. Pour cela nous avons créer un contrôleur EventManager dont le but est
uniquement de gérer les collisions. EventManager parcourt la liste de toutes
les entités en jeu et envoie un signal aux entités si il y a collision.
Cette partie a principalement été traitée par Guillaume.

\subsubsection{Ennemis}
Les ennemis sont représentés par la même classe que le joueur (Character), la
seule différence étant un booléen et leur déplacement automatisé.
Deux façons nous ont été proposées pour la gestion des ennemis, et notamment
pour leur déplacement. La première solution se basait sur un système de tour par
tour; le joueur se déplace, puis les ennemis. La deuxième solution était que
les ennemis se déplacent en temps réel, même si le joueur ne bougeait pas.
Nous avons opté pour la première solution, car nous la trouvions plus proche
de ce qui était attendu.
Nous avons donc implémenté l'algorithme de Manhattan afin que les ennemis
puissent se rapprocher du joueur. Cet algorithme permet de déterminer la
distance de chaque sommet à un sommet donné, ainsi on l'utilise principalement
pour déterminer la direction dans laquelle les ennemis devraient se déplacer
afin qu'ils se rapprochent du joueur. Le mouvement est effectué par le
GameManager.
Lorsque un ennemi touche un joueur, un signal est émis à l'eventHandler, qui
mettra fin au jeu.
Cette partie a principalement été faite par Romain.

\subsubsection{Boutons OnOff}
Après avoir implémenté les bases du jeu (joueur, ennemis, déplacements,
collisions), nous devions implémenté les portes qui s'ouvraient ou fermaient
à l'aide d'un bouton. Le bouton OnOff est également une entité qui hérite
de la classe abstraite Entity, cependant ses fonctionnalités sont plus
complexes que les autres entités telles que les bonbons. Pour se faire, nous
avons ajouté un membre à la classe OnOff, qui est une arête du graphe en modèle.
Ceci représente le mur qui est actionné par le bouton dans le labyrinthe. Lors
d'une collision entre le joueur et le bouton, un signal est envoyé à
l'EventManager, qui va envoyer un signal au modèle du graphe et modifier le
type de l'arête en question (WallType). Ceci aura pour effet d'ouvrir ou de
fermer une porte. La vue va ensuite mettre à jour l'affichage est modifier la
couleur du mur (vert/rouge). Cette partie a principalement été faite par Kenji.

\subsubsection{EntitySpawner}
Afin de facilier l'ajout/suppression d'entités dans notre labyrinthe, nous avons
créer une classe EntitySpawner dont le but est d'instancier une entité et de
la positionner sur le labyrinthe. Cela nous permet d'alléger le code dans la
classe Master et d'être plus efficace sur la gestion de nos entités.

%-------------------------------------------------------------------------------
% Bilan
%-------------------------------------------------------------------------------

\newpage
\section{Bilan}
Dans l'ensemble, nous trouvons que le projet s'est bien déroulé. Nous estimons
avoir répondu à la plupart des fonctionnalités attendues, même si certaines
paraissent "injustes" à cause de la génération complétement aléatoire des
positions.
La répartition des tâches n'a pas été simple, beaucoup de parties du projet
semblaient étroitement liées et nous avions du prendre le temps de mettre sur
papier un modèle MVC bien séparé afin de pouvoir travailler efficacement
en parrallèle.

%-------------------------------------------------------------------------------
% Perspective d'améliorations
%-------------------------------------------------------------------------------
\section{Perspective d'amélioration}

Une des améliorations que nous envisagions mais n'avons pas eu le temps
d'implémenter était une amélioration du contrôle de l'aléatoire. Dans la version
actuelle, toutes les entités sont générées à une position aléatoire dans le
labyrinthe. Cela signifie que le joueur peut très bien apparaître juste à côté
d'un ennemi. Une solution envisagée serait d'utiliser les valeurs obtenues par
l'algorithme de Manhattan afin de faire apparaître les ennemis à une distance
minimale du joueur. \\
Nous pensons également que certaines parties de notre code ne sont pas
pertinentes. Nous pouvons touver des commentaires "TODO : à déplacer" dans
certaines classes (ViewFrame par exemple), mais n'avons pas eu le temps de
nous concerter pour améliorer notre code.

%-------------------------------------------------------------------------------
% END
%-------------------------------------------------------------------------------

\end{document}
