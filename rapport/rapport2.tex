%-------------------------------------------------------------------------------
% COMPILATION
% pdflatex -shell-escape <fichier.tex>
%-------------------------------------------------------------------------------

\documentclass[a4paper]{article}

%% Language and font encodings
\usepackage[frenchb]{babel}
\usepackage[utf8x]{inputenc}
\usepackage[T1]{fontenc}
\usepackage{minted} %compiler avec la commande -shell-escape
\usepackage{graphicx}
\usepackage{titlesec}
\usepackage{hyperref}

%% Sets page size and margins
\usepackage[a4paper,top=3cm,bottom=2cm,left=3cm,right=3cm,marginparwidth=1.75cm]{geometry}
\setlength{\parskip}{.5em}

\newcommand{\HRule}{\rule{\linewidth}{0.5mm}}

%-------------------------------------------------------------------------------
% TITLE PAGE
%-------------------------------------------------------------------------------

\title
{
	\LARGE{Projet Approche Objet}
	\HRule \\ [0.5cm]
	\LARGE \textbf{\uppercase{Un jeu de plateau Labyrinthe}}
	\HRule \\ [0.5cm]
}

\author{Guillaume CHARLET \\ Kenji FONTAINE \\ Romain ORDONEZ \\ Adrien HALNAUT}

\begin{document}

\null  % Empty line
\nointerlineskip  % No skip for prev line
\vfill
\let\snewpage \newpage
\let\newpage \relax
\maketitle
\let \newpage \snewpage
\vfill
\break % page break

%-------------------------------------------------------------------------------
% Table of Contents
%-------------------------------------------------------------------------------

\tableofcontents
\newpage

%-------------------------------------------------------------------------------
% Introduction
%-------------------------------------------------------------------------------

\section{Description du projet}
Ce travail a été réalisé dans un cadre universitaire,  par des étudiants en
Master 1 informatique à l'université de Bordeaux.
Le but de ce projet est de mettre en pratique les connaissances acquises lors
du cours d'Approche Objet. \\
Labyrinth est un jeu de plateau dans lequel le joueur peut se déplacer et devra
trouver la sortie dans un labyrinthe en franchissant divers obstacles.

%-------------------------------------------------------------------------------
% Architecture du code
%-------------------------------------------------------------------------------

\section{Architecture du code}

Le dossier src contient notre code source pour le projet, réparti dans
différents sous-dossiers. On y trouve également Main, qui est éxécuté lorsque
le programme est lancé.

\subsection{Main}
Description de Main

\subsubsection{controllers}
\textbf{Master :} Description de Master.

\textbf{ingame :} \begin{itemize}
	\item{EntitySpawner : Instancie les entités du dossier drawable
	avec une position aléatoire.}
	\item{EventManager : S'occupe de détecter et de gérer les collisions entre
	entités (entre le joueur et un ennemi par exemple).}
	\item{GameManager : Gère le déroulement du jeu. Initialisation, gestion des
	déplacements des entités, ainsi que leur destruction (si un bonus est
	ramassé par exemple).}
	\item{Inputs : Reçoit les entrées utilisateurs afin de déplacer le joueur
	dans le labyrinthe.}
	\item{SpriteManager : Permet de récupérer les images utilisées dans
	l'interface graphique.}
\end{itemize}

\subsubsection{models}
\textbf{drawable :} \begin{itemize}
	\item{Bonus : Classe des bonbons ramassables par le joueur.}
	\item{Character : Classe utilisée par le joueur ainsi que les ennemis. On
	distingue le joueur des ennemis à l'aide d'un booléen.}
	\item{Door : Classe de la porte de sortie utilisée dans le labyrinthe.}
	\item{Entity : Classe abstraite héritée par toutes les autres entités. Elle
	définie le fonctionnement générale d'une entité.}
	\item{EntityType : Enumération des différentes entités pouvant être
	utilisées dans le jeu.}
	\item{OnOff : Classe des boutons permettant d'activer/désactiver un mur dans
	lebyrinthe.}
	\item{SpriteType : Enumérations des différentes images pouvant être
	associées à une entité dans le jeu.}
\end{itemize}

\textbf{game :} \begin{itemize}
	\item{maze/Direction : Enumération des directions différentes qu'un
	mouvement peut prendre.}
	\item{maze/Menu : Classe du Menu, non utilisée.}
	\item{maze/WallType : Enumération des différents états qu'un mur peut
	prendre.}
\end{itemize}

\subsubsection{views}
\textbf{ViewFrame :} Permet d'initialiser l'interface graphique ainsi que de
mettre l'affichage des entités à jour à chaque déplacement.

%-------------------------------------------------------------------------------
% Description du travail
%-------------------------------------------------------------------------------

\section*{} % Sinon compile pas
\section{Description du travail}

\subsection{Première séance}
Lors de la première séance, nous avons réfléchi sur la structure qu'allait
prendre notre code. Nous nous sommes mis d'accord pour partir sur une
architecture MVC (modèle, vue et contrôleur) afin de disposer d'une certaine
flexibilité. Nous avons ensuite déterminer sur papier, la hiérarchie de nos
classes, packages, etc... Puis nous nous sommes réparti le travail en groupe. \\
Voici le premier modèle retenu :
\begin{itemize}

	\item Model \begin{itemize}
		\item Drawable \begin{itemize}
			\item Gentil
			\item Mechant
		\end{itemize}
		\item Static \begin{itemize}
			\item Switches
			\item Doors
			\item Bonus
		\end{itemize}
		\item Game core \begin{itemize}
			\item Laby/Graph
			\item Menu
		\end{itemize}
	\end{itemize}

	\item View \begin{itemize}
		\item Rien pour l'instant
	\end{itemize}

	\item Controller \begin{itemize}
		\item Main/Master
		\item In-Game \begin{itemize}
			\item Inputs
			\item Events
			\item GameManager
		\end{itemize}
	\end{itemize}

\end{itemize}

\subsection{Premier pas}
Nous avons commencé notre implémentation par la génération de graphe (modèle)
ainsi que l'affichage de la grille (vue).

%-------------------------------------------------------------------------------
%
%-------------------------------------------------------------------------------

%-------------------------------------------------------------------------------
%
%-------------------------------------------------------------------------------

%-------------------------------------------------------------------------------
% END
%-------------------------------------------------------------------------------

\end{document}
